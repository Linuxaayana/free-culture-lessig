% This is -*- Latex -*-

% Also from https://titanpad.com/HztVHFmWft

\documentclass[a4paper, 11pt, norsk]{article}
\usepackage[utf8x]{inputenc}
\usepackage[norsk]{babel}
\usepackage{hyperref}
\usepackage{fullpage}

\parindent=0em
\setlength{\parskip}{0.5\baselineskip}%
\pagestyle{empty}

\begin{document}

\begin{flushright} Oktober 2015 \end{flushright}

{\bf Ærede Stortingsrepresentant,}

Boken «Fri kultur», som du nå holder i hånden, er resultatet av tre
års frivillig arbeide.  Idéen kom fra en diskusjon jeg hadde med en
venn for omtrent ti år siden, om opphavsrettsdebatten i Norge og hvor
sjelden de negative sidene ved økende regulering av åndsverk kom opp i
offentligheten.  Sommeren 2012 tok jeg endelig en ny titt på idéen og
bestemte meg for å gi ut boken «Free
  Culture»\footnote{\url{https://en.wikipedia.org/wiki/Free_Culture_(book)}}
på norsk, da denne beskriver problemstillingen veldig godt.  Den
norske utgaven er nærmere beskrevet på side 245 i boken.

Boken forteller hvordan store medieaktører ved hjelp av opphavsretten
bruker teknologi til å begrense kulturen og kontrollere kreativiteten.
Den er skrevet av stifteren av Creative
Commons\footnote{\url{https://creativecommons.org/}}, professor
Lawrence Lessig, som stiller til valg som presidentkandidat i USA i
2016.  Lessig ble sist omtalt i norske medier da NRK i høst viste
dokumentaren «The Internet's Own Boy: The Story of Aaron Swartz» som
også er tilgjengelig fra The Internet
Archive\footnote{\url{https://archive.org/details/TheInternetsOwnBoyTheStoryOfAaronSwartz}}.

Boken beskriver om hvordan opphavsrettens makt i USA har blitt
betydelig utvidet siden 1974 langs fem kritiske akser: varighet (fra
32 til 95 år), omfang (fra utgivere til alle), rekkevidde (gjelder nå
enhver fremvisning via datamaskin), kontroll (avledede verk er
definert så vidt at i praksis alt nytt innhold risikerer søksmål fra
en opphavsrettsinnehaver) samt maktkonsentrasjon og integrering av
mediebransjen.  Den dokumenterer også hvordan medieindustrien har
lyktes med å bruke rettsvesenet til å begrense konkurranse, og i
praksis skaffet seg vetorett over teknologiske nyvinninger. Nedlasting
av fritt, lovlig og i utgangspunktet gratis materiale stoppes med
tekniske sperrer og lobbyert lovvern av sperrene.

Når en vet hvordan opphavsrettens varighet i Norge, uten opposisjon på
Stortinget, ble utvidet i mai fjor, og hvordan Norges handelspartner
USA gjennom de nye handelsavtalene Trans-Pacific Partnership og
Transatlantic Trade and Investment Partnership ønsker å utvide
opphavsrettens makt\footnote{\url{https://www.eff.org/deeplinks/2015/10/final-leaked-tpp-text-all-we-feared}},
håper jeg flere vil spørre: Er det virkelig fornuftig å gjøre de samme
utvidelsene i Norge?

Denne boken er et bidrag til kunnskap og forståelse, og gir Stortinget
et bedre grunnlag til å ta riktige beslutninger som ivaretar
befolkningens og samfunnets interesser i Norge.

Selv er jeg en mangeårig fri programvareutvikler som har vært med på å
lage systemer som operativsystemet Debian, IT-løsningen Skolelinux,
borgerportalen FiksGataMi og innsynstjenesten Mimes brønn.  Jeg har
opplevd problemene med utvidet varighet, omfang, rekkevidde og
kontroll i opphavsretten på nært hold.

Jeg håper du finner boken like interessant som jeg gjorde.

NUUG Foundation har sponset trykkingen av dette eksemplaret.

Vennlig hilsen,

\vspace{4\parskip}

Petter Reinholdtsen \\
Oversetter og utgiver

\end{document}
